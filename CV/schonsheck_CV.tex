% Jason R. Blevins - Curriculum Vitae
%
% Copyright (C) 2004-2010 Jason R. Blevins
%this is a test
% You may use use this document as a template to create your own CV
% and you may redistribute the source code freely.  No attribution is
% required in any resulting documents.  I do ask that you please leave
% this notice and the above URL in the source code if you choose to
% redistribute this file.


\documentclass[10pt,letterpaper]{article}

\usepackage{hyperref}
\usepackage{geometry}
\usepackage{amsmath, amsfonts}
\usepackage{url}
\usepackage{etaremune}

% Fonts
%\usepackage[T1]{fontenc}
%\usepackage{garamondlibre}
%\usepackage{cmserif}
%\usepackage[urw-garamond]{mathdesign}




%\usepackage[lf,footnotefigures,minionint]{MinionPro}
%\usepackage[stretch=0, final]{microtype} % stretch 10 is okay, too
%\usepackage[T1]{fontenc}

% Set your name here
\def\name{Nikolas Schonsheck}

% The following metadata will show up in the PDF properties
\hypersetup{ colorlinks = true, urlcolor = black, pdfauthor = {\name},
  pdfkeywords = {mathematics, algebraic topology, homotopy theory,
    ring spectra}, pdftitle = {\name: Curriculum Vitae},
  pdfsubject = {Curriculum Vitae}, pdfpagemode = UseNone }

\geometry{
  body={6.5in, 9.0in},
  left=1.0in,
  top=.9in, bottom = 0.8in}

% Customize page headers
\pagestyle{myheadings}
\markright{\name}
\thispagestyle{empty}

% Custom section fonts
\usepackage{sectsty}
\sectionfont{\rmfamily\mdseries\bf\Large}
\subsectionfont{\rmfamily\mdseries\itshape\large}

% Other possible font commands include:
% \ttfamily for teletype,
% \sffamily for sans serif,
% \bfseries for bold,
% \scshape for small caps,
% \normalsize, \large, \Large, \LARGE sizes.

% Don't indent paragraphs.
\setlength\parindent{0em}

% Make lists without bullets and compact spacing
\renewenvironment{itemize}{
  \begin{list}{}{
    \setlength{\leftmargin}{1.5em}
    \setlength{\itemsep}{0.25em}
    \setlength{\parskip}{0pt}
    \setlength{\parsep}{0.25em}
  }
}{
  \end{list}
}

\begin{document}

% Place name at left
{\huge \name ~--- Curriculum Vit\ae}

% Alternatively, print name centered and bold:
%\centerline{\huge \bf \name}

\bigskip

\begin{minipage}[t]{0.5\textwidth}
  Center for Studies in Physics and Biology \\
  The Rockefeller University \\
  New York, NY 10065
\end{minipage} \hfill
\begin{minipage}[t]{0.5\textwidth}
  Office: Smith Hall Annex B17 \\
%  Mobile: (617) 875-1988 \\
  Email: \url{nschonsheck@rockefeller.edu} \\
  Website:
  \href{https://niko-schonsheck.github.io/}{https://niko-schonsheck.github.io/}
\end{minipage}



\section*{Education \& Employment}


\begin{itemize}

\item {\bf The Rockefeller University}
\begin{itemize}
\item Independent Fellow \hfill August 2024-present

\vspace{-1ex}

\item Center for Studies in Physics and Biology
\end{itemize}





\item {\bf University of Delaware}
\begin{itemize}
\item Postdoctoral Fellow \hfill July 2021-July 2024

\vspace{-1ex}

\item Department of Mathematical Sciences

\vspace{-1ex}

\item PI: Chad Giusti
\end{itemize}

\item {\bf The Ohio State University}
	
	\begin{itemize}
	\item Ph.D. in Mathematics \hfill May 2021
	
%	\vspace{-1ex}
	
%	\item M.S. in Mathematics \hfill December 2018
	
	\vspace{-1ex}
		
	\item Supervisor: John E. Harper
	\end{itemize}

 
\item {\bf Vassar College}
  
  \begin{itemize}
	\item B.A. in Mathematics, general and subject honors \hfill May 2015
	
	\vspace{-1ex}
	
  	\item Minor in social and political philosophy 
  \end{itemize}

\end{itemize}



\section*{Publications \& Preprints}

\begin{etaremune}\itemsep.7em


\item{\bf Topological data analysis of circulant structure in neural architecture and function} (with N. Sanderson). In preparation.


\item{\bf Learning local geometry and nonlinear topology of neural manifolds via spike-timing dependent plasticity} (with C. Giusti). Available at \url{https://www.biorxiv.org/content/10.1101/2025.08.27.672728v1}.

\item {\bf Dowker's theorem for higher order relations} (with C. Giusti, V. Itskov, M. Robinson, R. Sazdanovic, V. de Silva, M. Vaupel, H-R. Yoon). Submitted. Available at \url{https://arxiv.org/abs/2506.10909}.


\item{\bf $O(k)$-equivariant dimensionality reduction on Stiefel manifolds} (with A. Lee, H. Lee, J. Perea, M. Weinstein). \emph{SIAM Journal on the Mathematics of Data Science}, \textbf{7}(2), 2025. Available at \url{https://arxiv.org/pdf/2309.10775.pdf}.

\item {\bf Spherical coordinates from persistent cohomology} (with S. Schonsheck). \emph{Journal of Applied and Computational Topology}, \textbf{8}, 149-173 (2024). Available at \url{https://arxiv.org/pdf/2209.02791.pdf}.

\item{\bf Toroidal coordinates: decorrelating circular coordinates with lattice reduction} (with J. Bush, H. Gakhar, J. Perea, T. Rask, L. Scoccola, L. Zhou). \textit{$39^{th}$ International Symposium on Computational Geometry (SoCG 2023)}, \textbf{258},  57:1-57:20 (2023). Available at \url{https://arxiv.org/abs/2212.07201}.

\item{\bf On the chromatic localization of the homotopy completion tower for $O$-algebras} (with C. Ogle). \textit{New York Journal of Mathematics}, {\bf 28}, 1042-1056 (2022). 

%Also available at \url{https://niko-schonsheck.github.io/research_files/E_n_localization.pdf}/

%ON THE CHROMATIC LOCALIZATION OF THE HOMOTOPY COMPLETION TOWER FOR O-ALGEBRAS

\item{\bf $\mathsf{TQ}$-completion and the Taylor tower of the identity functor}. \textit{Journal of Homotopy and Related Structures}, \textbf{17}, 201-216 (2022). 

%Also available at \url{https://people.math.osu.edu/schonsheck.2/research_files/Taylor_tower_of_identity.pdf}.


\item {\bf Fibration theorems for $\mathsf{TQ}$-completion of structured ring spectra}. \textit{Tbilisi Math. Journal: Special Issue on Homotopy Theory, Spectra, and Structured Ring Spectra}, 1-15 (2020). 

%Also available at \url{https://people.math.osu.edu/schonsheck.2/research_files/Fibration_theorems.pdf}.

\item {\bf On the cop number of generalized Petersen graphs} (with T. Ball, R. Bell, J. Guzman, and M. Hanson-Colvin). \textit{Discrete Mathematics}, \textbf{340} (6), 1381-1388 (2017). 









%Undergraduate research from Michigan State University REU. Available at \url{https://www.sciencedirect.com/science/article/abs/pii/S0012365X16303387/}.
\end{etaremune}

% \section*{Publications and preprints}



\section*{Teaching Experience}

\subsection*{University of Delaware}
\begin{itemize}
\item {\bf {\fontsize{11}{11}\selectfont Instructor}}\\
{Math 367: Seminar on Applied Topology \hfill Spring 2023}

{Math 349: Elementary Linear Algebra \hfill Fall 2022}

{Math 401: Introduction to Analysis} \hfill Spring 2022
\end{itemize}



\subsection*{The Ohio State University}
\begin{itemize}
\item {\bf {\fontsize{11}{11}\selectfont Instructor}}\\
{Math 1149: Trigonometry} \hfill Summer 2018




\item {\bf {\fontsize{11}{11}\selectfont Co-instructor}}\\
{Math 1125: Mathematics for Elementary Teachers I} \hfill Fall 2019



Math 1149: Trigonometry \hfill Summer 2017\\




\item {\bf {\fontsize{11}{11}\selectfont Graduate Teaching Associate}} \\
  {Regular duties included planning material for and conducting two recitation meetings per week, writing and grading quizzes, grading homework.}
  
  \begin{itemize}
  \item Math 1150: Precalculus
    \hfill Fall 2020
  
  \item Math 1161: Accelerated Calculus I 
  \hfill Fall 2018, Fall 2017
  
  \item Math 1151: Calculus I
  \hfill Spring 2018, Fall 2016 \\
  \vphantom{M}\hfill Spring 2016, Fall 2015
  
  \item Math 1152: Calculus II
  \hfill Spring 2017

  \end{itemize}

\subsection*{Vassar College}
\begin{itemize}
\item {\bf Undergraduate Assistant} \hfill Fall 2014-Spring 2015 \\
  {Held six office hours per week for upper-level mathematics classes.}
\end{itemize}

\end{itemize}



\section*{Research Mentoring and Service}
\begin{itemize}

	\item {\bf Summer Scholars Program} \hfill Summer 2023-Spring 2024
	
	\begin{itemize}
	\item Undergraduate summer research program at University of Delaware. Supervising two undergraduate students applying persistent cohomological techniques to artificial neural networks and Hebbian learning rules.
	\end{itemize}

	

	\item {\bf GEMS Summer Program} \hfill Summer 2022
	
	\begin{itemize}
	Summer research program at University of Delaware; supervisor to one undergraduate and one graduate student studying propagation of cyclical data features through feedforward neural networks.
	\end{itemize}

	\item {\bf Directed Reading Program} (University of Delaware) \hfill Spring 2022-present
	
	\begin{itemize}
	\item Co-founded and continue to organize Directed Reading Program at University of Delaware; supervised reading projects on simplicial homology, elementary number theory, and combinatorial pursuit games.
	\end{itemize}
	\item {\bf Knots and Graphs undergraduate research working group} \hfill Summer 2020
	
	\begin{itemize}
		\item Summer research program similar to an REU but only open to Ohio State students; volunteered to mentor two groups of four undergraduate students working on problems in graph coloring.
	\end{itemize}

	
	\item {\bf Directed Reading Program} (Ohio State) \hfill Spring 2019

	
	\begin{itemize}
		\item Oversaw a reading course on introductory algebraic topology while a graduate student at Ohio State.
	\end{itemize}
\end{itemize}







\section*{Honors, Awards, \& Fellowships}




%JMM Poster session award?

\subsection*{Teaching}
%Finalist for the math department one?
\begin{itemize}

\item {\bf Excellence in Teaching Award (Nominated)} \hfill Spring 2023\\
University of Delaware, independently nominated for ``excellent work and your positive impact on student learning.''

\item{\bf Phil Huneke Distinguished Graduate Teaching Associate Award} \hfill Spring 2021\\
Departmental. This award recognizes ``mathematics graduate students who have demonstrated excellence in the classroom and a high level of commitment to undergraduate mathematics education.'' Awarded for the 2019-2020 academic year, but awards delayed to 2021 due to the COVID-19 pandemic.



\item {\bf Graduate Associate Teaching Award} \hfill Spring 2020\\
  % May 5, 2014
 University-wide. ``Ohio State's highest recognition of teaching done by graduate students.'' Ten recipients chosen each year out of over 3,000 graduate TA's.
\item {\bf First-year Graduate Teaching Associate Award} \hfill Spring 2016 \\
Departmental. ``This award recognizes outstanding first year Graduate Teaching Associates within the OSU Department of Mathematics.''

\end{itemize}

\subsection*{Research \& Scholarship}
\begin{itemize}
\item {\bf Research Training Groups (RTG) Fellowship} \hfill
  Summer 2020, Spring 2020 \\
  {Department of Mathematics, The Ohio State University} \hfill Spring 2019

\item {\bf Mary Evelyn Wells and Gertrude Smith Prize} \hfill Spring 2011\\
{Department of Mathematics, Vassar College}


\end{itemize}





\section*{Other Service and Activities}

\begin{itemize}

\item{\bf Minisymposium on Topology and Geometry in Neuroscience} \hfill July 2025\\
Co-organized minisyposium, SIAM Applied Algebraic Geometry Conference


\item{\bf Computational Systems Neuroscience 2025} \hfill March 2025\\
Reviewer for COSYNE 2025.

\item{\bf JMM Special Session} \hfill January 2024\\{\bf Applied Topology Beyond Persistence Diagrams}\\
{Co-organized special session on applied topology focusing on applications of homological algebra beyond the industry standard of persistence diagrams.}




\item{\bf JMM Special Session} \hfill January 2023\\{\bf Applied Topology: Theory and Implementation}\\
{Co-organized special session on applied topology with a particular view towards bridging the gap between the theory and implementation of recent research in applied topology and topological data analysis.}
\item {\bf AMS Mathematics Research Communities} \hfill Summer 2022\\
{Participant in MRC: Data Science at the Crossroads of Analysis, Geometry, and Topology. Worked with two groups of other early career researchers on projects in topological data analysis. Projects are ongoing and in preparation to submit for publication.}
\item {\bf Addressing Barriers to Participation in STEM} \hfill Fall 2021-present\\
{Member, committee of University of Delaware Anti-Racism Initiative. Activities have included developing materials for holistic admission processes, lobbying for required diversity and inclusion questions in faculty hiring, and successfully advocating the raise of minimum graduate student stipend.}
\item{\bf Invited Panelist, AWM Chapter at Marian University} \hfill Spring 2021\\
{Served as a panelist for a discussion on transitioni ng from an undergraduate liberal arts school to graduate school/industry in STEM fields.}
\item {\bf Buckeye Aha! Math Moments} \hfill Summer 2020 \\
  {Volunteered to mentor and review student work for this outreach initiative of the Department of Mathematics at OSU.}
\item {\bf Mentor for TA training}
  \hfill Summer 2020, 2019, 2018, 2017 \\
  {Assisted in summer training of incoming TA's at Ohio State. }
\item {\bf TA Peer mentor} \hfill Fall 2019, 2018, 2017, 2016 \\
  {Served in the peer-mentoring program for new TA's at Ohio state; oversaw a total of 14 new teaching associates.}

\end{itemize}

\section*{Selected Research Talks}


\subsection*{Invited and contributed talks}
\begin{itemize}

\item{\bf EPFL Topology Seminar} \hfill September 2025\\
``Circles in the brain: learning local geometry\\and nonlinear topology via spike-timing dependent plasticity''

\item{\bf Workshop on Topological Data Analysis} \hfill August 2025\\
{\bf Fields Institute, Toronto, Ontario} \\
``Tutorial: Topological and Geometric Methods in Neuroscience''

\item{\bf SIAM Conference on Applied Algebraic Geometry} \hfill July 2025\\
{\bf Madison, Wisconsin}\\
``Minitutorial on Topological and Geometric Methods in Neuroscience''\\
``Learning Circular Coordinate Systems via Spike Timing Dependent Plasticity''

\item{\bf Vassar College Colloquium} \hfill December 2024\\
``An introduction to applied algebraic topology''


\item{\bf SIAM Conference on Mathematics of Data Science} \hfill October 2024\\
{\bf Atlanta, Georgia}\\
``Topological dimensionality reduction via persistent cohomology''\\
``Hebbian learning of cyclic features of neural code'' (poster)


\item{\bf International Conference on Mathematical Neuroscience} \hfill June 2024\\
{\bf University College Dublin}\\
``Relative neural population size modulates learnability of cyclic features of neural code''

\item{\bf NEXTEN Conference, Washington University in St. Louis} \hfill May 2024\\
{\bf Washington University in St. Louis}\\
``Relative neural population size modulates learnability of cyclic features of neural code''\\
(Poster presentation.)


\item {\bf JMM Special Session on Applied Topology: Theory, Algorithms,} \hfill January 2024\\
{\bf and Applications, San Francisco} \\
``Spherical parameterizations of data via persistent cohomology''


\item {\bf AMS Sectional Special Session on Discrete, Algebraic,}\hfill October 2023 \\
{\bf and Topological Methods in Mathematical Biology, Creighton University} \\
``Hebbian learning of cyclic structures in neural coding''


\item {\bf Computational Neuroscience Annual Meeting, Leipzig, Germany} \hfill July 2023\\
``Relative neural population size modulates learnability of cyclic features of neural code''\\
(Poster presentation)

\item {\bf Joint Mathematics Meetings, MRC Special Session} \hfill January 2023\\
``Equivariant dimensionality reduction on Stiefel manifolds''


\item {\bf Geometry-Topology Seminar, Oregon State University} \hfill October 2022\\
``Spherical coordinates from persistent cohomology''



\item {\bf Topology Seminar, University of Iowa} \hfill April 2021\\
``Fibration theorems, functor calculus, and chromatic connections in $O$-algebras''


\item{\bf Graduate Student Topology and Geometry Conference, Indiana University} \hfill April 2021\\
``Functor calculus and chromatic connections in $O$-algebras''

\item {\bf Topology Seminar, University of Regina} \hfill January 2021\\
``Fibration theorems, functor calculus, and chromatic connections in $O$-algebras''

\item {\bf Graduate Conference in Algebra and Topology, Binghamton University} \hfill November 2020\\
``$\mathsf{TQ}$-completion: fibration theorems and connections to functor calculus''

\item {\bf Algebraic Topology Seminar, UCLA} \hfill October 2020\\
``Fibration theorems and functor calculus for structured ring spectra''

\item{\bf Topology Seminar, Pennsylvania State University-Altoona} \hfill September 2020\\
``$\mathsf{TQ}$-completion: fibration theorems and connections to functor calculus''


\item{\bf Topology Seminar, University of Virginia} \hfill September 2020\\
``$\mathsf{TQ}$-completion: fibration theorems and connections to functor calculus''
	
	
\item {\bf AMS Sectional Special Session on Homotopy Theory, University of Virginia} \hfill March 2020\\
``$\mathsf{TQ}$-completion of certain fibration sequences''\\
(This conference was canceled due to the COVID-19 pandemic; \\notes available at \url{http://people.virginia.edu/~jeb2md/Schonsheck2020.pdf})

\item {\bf Young Topologists Meeting, EPFL, Switzerland} \hfill July 2019 \\
``Topological Quillen homology of structured ring spectra''
 
 
\item {\bf Mathematics Colloquium, Vassar College} \hfill February 2019 \\
``Homotopy theory---from the fundamental group to structured ring spectra''

  \item {\bf Young Topologists Meeting, University of Copenhagen} \hfill July 2018 \\
 ``An introduction to symmetric spectra''

\end{itemize}

\subsection*{Informal talks}
\begin{itemize}
	
\item {\bf GOATS 2 Online Mini-Conference} \hfill {June 2020}\\
	``Fibration Theorems for $\mathsf{TQ}$-completion of structured ring spectra''\\
	(Available at \url{https://youtu.be/NZ71N1-CUZQ})\\
	
	
\item {\bf GROOT Summer Seminar, online} \hfill May 2020\\
``Fibration Theorems for $\mathsf{TQ}$-completion of structured ring spectra''\\
(Available at \url{https://youtu.be/DkjCgY1kjF8} and \url{https://youtu.be/EUAh8fwjF9M})\\

\item{\bf Student Homotopy Seminar, Ohio State Mathematics Department}\hfill 2018-2020\\
``Pro-nilpotent homology types''\\
``Fibration theorems for $\mathsf{TQ}$-completion of structured ring spectra''\\
``Long homology localization towers''\\
``Localization and completion with respect to topological Quillen homology''\\
``Cosimplicial resolution model structures''\\
``The role of principal fibrations''\\
``Completion of spaces and ring spectra with respect to homology''\\
``Operads and the recognition principle''\\
``Comparing $H\mathbb{Z}$-algebras in $Sp^\Sigma$ to unbounded chain complexes''\\
``Why symmetric spectra?''\\

\item{\bf Graduate Student Seminar, Ohio State Mathematics Department} \hfill January 2019\\
``Homotopy theory---from the fundamental group to structured ring spectra''\\


\item{\bf Seminar-$\infty$, Ohio State Mathematics Department} \hfill Fall 2017\\
``The Dold-Kan Correspondence''\\
``Eilenberg-Zilber and geometric realization''\\


\item {\bf Various Presentations} \hfill 2014-2015\\
``The game of Cops and Robbers on graphs''

\end{itemize}





\subsection*{Conference participation}
\begin{itemize}

	\item{\bf Fields Institute Workshop on Topological Data Analysis} \hfill August 2025

	\item{\bf BIRS Workshop: Cycle Representatives in Applied Homological Algebra} \hfill August 2025

	\item{\bf SIAM Conference on Applied Algebraic Geometry} \hfill July 2025
	
	\item{\bf COSYNE Computational Systems Neuroscience} \hfill March 2025

	\item{\bf SIAM Conference on Mathematics of Data Science} \hfill October 2024

	\item{\bf International Conference on Mathematical Neuroscience} \hfill June 2024
	
	\item{\bf NEXTEN Conference, Washington University in St. Louis} \hfill May 2024

	{\item {\bf Joint Mathematics Meetings, San Francisco, CA} \hfill January 2024}
	
	{\item {\bf Neural Coding and Combinatorics, ICERM, Brown University} \hfill November 2023}
	
	{\item {\bf Topology and Geometry in Neuroscience, ICERM, Brown University} \hfill October 2023}
	
	{\item {\bf AMS Central Sectional Meeting, Creighton University} \hfill October 2023}

	\item {\bf Computational Neuroscience 2023} \hfill July 2023\\
	{\bf Annual meeting, Leipzig, Germany}


	\item {\bf Applied Homological Algebra Beyond Persistence Diagrams} \hfill June 2023\\
	{\bf Workshop, American Institute for Mathematics}
	
	{\item {\bf Joint Mathematics Meetings, Boston MA} \hfill January 2023}
	
	\item {\bf Algebraic Topology and Topological Data Analysis:} \hfill August 2022\\
	{\bf A Conference in Honor of Gunnar Carlsson}\\
	{\bf Institute for Mathematics and its Applications, University of Minnesota} 
	\item {\bf Mathematics Research Communities:} \hfill June 2022\\ {\bf Data Science at the Crossroads of Analysis, Geometry, and Topology}\\
	{\bf American Mathematical Society, Beaver Hollow, NY}
	\item {\bf Hot Topics Workshop:} \hfill May 2022\\{\bf Topological and Dynamical Analysis of Brain Connectomes} \\
	{\bf ICERM, Brown University}
	\item {\bf COSYNE: Computational Systems Neuroscience, Lisbon, Portugal} \hfill March 2022
	\item {\bf Graduate Student Topology and Geometry Conference, Indiana University} \hfill April 2021
	\item {\bf Graduate Conference in Algebra and Topology, Binghamton University} \hfill November 2020
	\item {\bf Midwest Topology Seminar, Virtual, Wayne State University} \hfill October 2020
	\item {\bf GOATS 2 Online Mini-Conference} \hfill June 2020
	\item {\bf AMS Sectional Meeting, University of Wisconsin-Madison} \hfill September 2019
	\item {\bf Young Topologists Meeting, EPFL, Switzerland} \hfill July 2019
	\item {\bf Midwest Topology Seminar, Michigan State University} \hfill May 2019
	\item {\bf Graduate Student Topology and Geometry Conference, UIUC} \hfill March 2019
	\item {\bf Functor Calculus Workshop, Ohio State University} \hfill March 2019
	\item {\bf Midwest Topology Seminar, University of Kentucky} \hfill September 2018
	\item {\bf Young Topologists Meeting, University of Copenhagen} \hfill July 2018
	\item {\bf Midwest Topology Seminar, Indiana University} \hfill April 2018
	\item {\bf AMS Sectional Meeting, Ohio State University} \hfill March 2018
	\item {\bf Midwest Topology Seminar, Northwestern University} \hfill March 2018
	\item {\bf Midwest Topology Seminar, Wayne State University} \hfill November 2017
	\item {\bf Homotopy Theory: Tools and Applications, UIUC} \hfill July 2017
	
\end{itemize}







\vfill
% Footer
\begin{center}
  \begin{small}
    Last updated: \today
  \end{small}
\end{center}

\end{document}
